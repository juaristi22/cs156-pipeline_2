\begin{abstract}
Policy microsimulation requires comprehensive microdata combining demographic, income, and wealth information, yet such data rarely coexist in a single survey. Statistical matching addresses this gap by imputing variables from a donor survey onto a receiver survey. We investigate whether deep generative models can improve upon Quantile Random Forests (QRF), the current state-of-the-art for imputing net worth from the Survey of Consumer Finances (SCF) onto the Current Population Survey (CPS). We evaluate three deep learning architectures: RealNVP (normalizing flows), Mixture Density Networks (MDN), and TabSyn (VAE with latent diffusion). Using 3-fold cross-validation with quantile loss and distributional accuracy metrics (Wasserstein distance and Kolmogorov-Smirnov statistic), only MDN demonstrates competitive performance. MDN achieves 16\% lower median quantile loss than QRF ($2.11 \pm 0.07$ vs. $2.51 \pm 0.01$), though QRF produces better distributional accuracy with a Wasserstein distance of 472,000 compared to MDN's 1.2 million. RealNVP and TabSyn prove unsuitable in their current configurations, with Wasserstein distances exceeding 12 million, generating distributions that fail to preserve the characteristic shape of wealth data. These findings suggest that while deep generative models show promise for survey imputation, careful architecture selection and hyperparameter tuning are essential, and traditional ensemble methods remain strong baselines for practitioners.\footnote{Code is available at \url{https://github.com/juaristi22/cs156-pipeline_2}.}
\end{abstract}
